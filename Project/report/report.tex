\documentclass{scrartcl}

\usepackage[table,xcdraw]{xcolor}
\usepackage{listings}
\usepackage[T1]{fontenc}
\usepackage[utf8]{inputenc}
\usepackage{graphicx}
\usepackage{hyperref}
\usepackage{float}
\usepackage{amsmath}
\usepackage[headsepline]{scrlayer-scrpage}
\usepackage{tikz}
\usepackage{siunitx}
\usepackage[a4paper,bottom=25mm]{geometry}


\begin{document}

\input{title}

\pagenumbering{arabic}
\pagestyle{scrheadings}
\clearscrheadfoot

\ihead{SoRTES Project}
\cfoot{\pagemark}

\newpage

\section{The system}
The system consists of 4 tasks:
\begin{itemize}
    \item LoRaReceiver
    \item DatabaseController
    \item LoRaTransmitter
    \item CommandManager
\end{itemize}

There are 2 queues, the \textit{DatabaseQueue} and the \textit{LoRaTransmitterQueue}.
The first queue is used for communication from \textit{LoRaReceiver} to the \textit{DatabaseController} and the
other one is used for communication from the \textit{DatabaseController} to the \textit{LoRaTransmitter}.
The \textit{LoRaReceiver} has the highest priority, since this is the most important task.
The \textit{DatabaseController} has the same priority, because they have to run at the same time in the beginning.
The other two tasks both have the same priority but it's one priority lower then the \textit{LoRaReceiver}.
There is also a semaphore that is required for every access to the database.
The database is implemented as a circular buffer using every address of the EEPROM.
In the database we store a pair that contains the time until the next beacon and the temperature read from the sensor.

\subsection{LoRaReceiver}
This task receives and reads incoming beacons. Packets with a size smaller than 5 will be ignored because
we assume that the message of the beacon has a length of at least 5 characters. The system can handle packets with more characters
as long as they follow the format, that the first 4 characters are the gateway ID and the remaining characters are the time until the
next beacon transmission in seconds. Once a beacon is received, the time $t$ until the next beacon is put in the \textit{DatabaseQueue} and
the \textit{firstPackageReceived} flag is set to \textit{true}.
Next the \textit{DatabaseController} task is resumed and the task will go to sleep for $t*1000 - \textit{delayBuffer}$ \si{\milli\second}.
We set the \textit{delayBuffer} to \SI{200}{\milli\second} to make sure we don't miss the next beacon.
After resuming the task, the LoRa module and ADC is turned on again.
When the task processed 20 beacons, it will put the system in ultra low-power mode.

\subsection{DatabaseController}
This task makes data packages and stores them in the database.
In the loop, the task reads values from the \textit{DatabaseQueue} and runs as long as there are items available in the queue.
If the queue is empty, the task will suspend itself and only resume when the function \textit{vTaskResume()} is called in the \textit{LoRaReceiver}.
The item in the queue gets dequeued. Next the task attempts to take the semaphore and waits for the semaphore if it isn't available.
We get the temperature from the temperature sensor and then we write the temperature with the data from the queue to the database (EEPROM).
After storing the data in the database, the semaphore is returned.
The last step is putting the temperature into the \textit{LoRaTransmitterQueue} and resuming the \textit{LoRaTransmitter} task.
Now we have processed one item from the \textit{DatabaseQueue} and we can start the loop again if more items are available in the queue.

\newpage

\subsection{LoRaTransmitter}
This task reads values from the \textit{LoRaTransmitterQueue} and runs as long as there are items available in the queue.
If the queue is empty, the task will suspend itself and only resume when the function \textit{vTaskResume()} is called in the \textit{DatabaseController}.
This task reads the temperature value from the queue, makes a packet with the temperature as message and sends it back to the gateway (GW).

\subsection{CommandManager}
This task reads commands from the serial port and prints output to the serial port at a rate of 9600 baud.
There are 4 commands supported (the input for these functions is just the numbers 1, 2, 3 and 4):
\begin{enumerate}
    \item Read the latest temperature value and beacon details from database and print the output to the serial port
    \item Read all temperature values and beacon details from database and print the output to the serial port
        (the oldest value will get printed first and the most recent value last)
    \item Turn on ultra low-power mode
    \item Reset the database (this should be done after flashing the program for the first time)
\end{enumerate}
When the first package is received, the task removes itself as a task and disables the usb interface to save power.
For command 1, 2 and 4, the task has to take the semaphore because it has to access the database.

\subsection{The Database}
The database is a circular buffer that uses every address of the EEPROM.
The first 4 bytes of the EEPROM are reserved and contain an address and the amount of data in the database.
If the database gets full, we start overwriting the oldest data first.
We reset the database by writing zero to the first 4 bytes of the EEPROM.
Every access to the database requires taking the semaphore, to keep the data consistent.

\section{Varying real-time constraint}
The task \textit{LoRaReceiver} will sleep for $t*1000 - \textit{delayBuffer}$ \si{\milli\second} where $t$ is the time in the received beacon.
The \textit{delayBuffer} is set to \SI{200}{\milli\second} to prevent missing beacons.
When the task wakes up again, it resumes the LoRa module (because it goes into sleep mode when the system is in low-power operation mode)
and is ready again to accept beacons. $t$ is not the duration that the whole system will be in low-power operation mode.
The system only goes in low-power operation mode when all other tasks are finished.
We can fine tune the \textit{delayBuffer} to be more power effcient but this requires detailed knowledge of the exact time
we receive beacons.

\newpage

\section{Synchronization}
We use a semaphore for accessing the database. Only the \textit{DatabaseController} and the \textit{CommandManager} acquire the semaphore.
The semaphore will always get returned by the task before it gets suspended or deleted so tasks won't get stuck waiting on a semaphore
that isn't going to be released.

\section{Power consumption}
\subsection{Low-power operation mode}
When there are no tasks active, the system goes into the \textit{vApplicationIdleHook}.
Here we turn off the ADC (Analog to Digital Converter),
put the LoRa module in sleep mode and the microcontroller (MCU) in Idle mode.
This leads to a power consumption of $\sim \SI{2.90}{\milli\ampere}$.
Remember that the usb interface is also turned off.
We leave the low-power operation mode as soon as the \textit{LoRaReceiver} task gets active again and
turn on the LoRa module and ADC.

\subsection{Ultra low-power mode}
After receiving 20 beacons or after executing the third command, the system goes into ultra low-power mode.
First we remove the task scheduler and disable the LoRa module.
Next we disable the usb interface (if the third command was used), turn off ADC, disable WDT and set all pins to output and low.
We put the MCU in power down mode and use the function \textit{power\_all\_disable()} to disable all modules.
This function from the AVR library will set the bits for us in the power registers so that everything is turned off.
Here we have a power consumption of $\sim \SI{820}{\micro\ampere}$.


\end{document}
